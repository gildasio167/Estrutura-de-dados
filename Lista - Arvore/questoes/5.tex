Apresente o pseudocódigo de uma função ImprimeNos(r) \textbf{não-recursiva} que recebe como
entrada o nó raiz r de uma BST e imprime os nós desta BST em ordem simétrica. \\
Dica: utilize as funções MinimoBST(r) (que retorna o nó com a menor chave na árvore) e
SucessorBST(v) (que retorna o nó sucessor de v).

\begin{itemize}
	\item \textbf{Resposta:}
	
	\begin{itemize}
		\item \textbf{Algoritmo:} imprimir\_interativo(\textbf{NO*} arv)
		\item \textbf{Entrada:} NO raiz da arvore
		\item \textbf{Saída:}
		Imprimir os NOs da arvore
		\item \textbf{Complexidade:} O(n)
		
		\begin{enumerate}[1--]
		
		\item \textbf{NO*} aux = \textbf{minimo}(arv)
		\item \textbf{enquanto} aux $\neq$ NULL \textbf{faça}
		\item $|$ \quad
		cout $<<$ aux $\rightarrow$ chave $<<$ endl
		\item $|$ \quad
		aux = \textbf{sucessor}(aux)
		
		\end{enumerate}
	\end{itemize}
\end{itemize}